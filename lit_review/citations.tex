\documentclass[12pt]{article}
\usepackage[margin=1in]{geometry}
\usepackage{parskip}
\usepackage{titlesec}
\usepackage{enumitem}
\usepackage{hyperref}
\usepackage{url}
\titleformat*{\section}{\large\bfseries}
\titlespacing*{\section}{0pt}{1.5em}{0.5em}
\begin{document}
\section*{Full Citation List by Paper}

\subsection*{Kemble 2003}
\begin{enumerate}
\item Cornelisse, J.W., Schoyer, H.F.R., Wakker, K.F., Rocket Propulsion and Spacecraft Dynamics, Pitman, 1979.
\item Jewitt, J.C., Sheppard, S., Porco, P., ``Jupiter’s Outer Satellites and Trojans,'' invited review for Jupiter (ed. F. Bagenal), Cambridge University Press, 2004.
\item Carusi, A., Valsecchi, G., ``Numerical simulation of close encounters between Jupiter and minor bodies,'' Asteroids (Univ. of Arizona Press, Tucson), pp. 391--416, 1979.
\item Koon, W.S., Lo, M.W., Marsden, J.E., Ross, S.D., ``Resonance and capture of Jupiter comets,'' Celestial Mechanics and Dynamical Astronomy 81(1--2), 27--38 (2001).
\item Ross, S.D., ``Statistical theory of interior--exterior transition and collision probabilities for minor bodies in the solar system,'' Libration Point Orbits and Applications (eds. G. Gómez, M.W. Lo, J.J. Masdemont), World Scientific, 2003, pp. 637--652.
\item Gómez, G., Koon, W.S., Lo, M.W., Marsden, J.E., Masdemont, J., Ross, S.D., ``Invariant manifolds, the spatial three-body problem and space mission design,'' AAS/AIAA Astrodynamics Specialist Conference, Québec City, Canada, 2001.
\item Kemble, S., Taylor, M.J., ``Mission design options for a small satellite mission to Jupiter,'' IAF-03-A.09, Proc. IAC Bremen, 2003.
\item Langevin, Y., ``Chemical and solar electric propulsion options for a Mercury cornerstone mission,'' IAF-99-A.2.04, Proc. 50th IAF Congress, Amsterdam, Oct. 1999.
\item Casalino, L., Colasurdo, G., Pastrone, D., ``Optimal low-thrust escape trajectories using gravity assist,'' Journal of Guidance, Control, and Dynamics 22(5), 1999.
\item Kemble, S., ``Optimised Transfers to Mercury,'' IAF-01-A.5.03, Proc. IAC Toulouse, 2001.
\end{enumerate}

\subsection*{Melman et al. 2008}
\begin{enumerate}
\item Carroll, W. and Ostlie, A., An Introduction to Modern Astrophysics, Addison-Wesley, Reading, 1996.
\item Noca, M. and Bailey, R., ``Mission Trades for Aerocapture at Neptune,'' AIAA-2004-3843, 40th AIAA/ASME/SAE/ASEE Joint Propulsion Conference, Fort Lauderdale, FL, July 2004.
\item Gooding, R., ``A Procedure for the Solution of Lambert’s Orbital Boundary-Value Problem,'' Celestial Mechanics and Dynamical Astronomy 48, 1990, pp. 145--165.
\item Cornelisse, J., Schöyer, H., and Wakker, K., Rocket Propulsion and Spaceflight Dynamics, Pitman Publishing, London, 1979.
\item Sponnick, J. and Jensen, M., ``Atlas Launch System Mission Planner’s Guide,'' Tech. Rep., International Launch Services, 2004 (Rev. 10).
\item Larson, W. and Wertz, J., Space Mission Analysis and Design, Microcosm Press, Torrance, 3rd ed., 1999.
\item Miner, E.D. and Wessen, R.R., Neptune: The Planet, Rings and Satellites, Springer-Praxis, New York, 2002.
\item Vasile, M., ``Robust Mission Design Through Evidence Theory and Multiagent Collaborative Search,'' Annals of the New York Academy of Sciences 1065, 2005, pp. 152--173.
\item Regan, F.J. and Anandakrishnan, S.M., Dynamics of Atmospheric Re-entry, AIAA Education Series, Washington, DC, 1993.
\item Lockwood, M.K., ``Neptune Aerocapture Systems Analysis,'' AIAA-2004-4951, AIAA Atmospheric Flight Mechanics Conference, Providence, RI, Aug. 2004.
\item Campagnola, S. and Lo, M., ``BepiColombo Gravitational Capture and the Elliptic Restricted Three-Body Problem,'' Proceedings in Applied Mathematics and Mechanics, Nov. 2007.
\item Russell, R. and Lam, T., ``Designing Ephemeris Capture Trajectories at Europa Using Unstable Periodic Orbits,'' Journal of Guidance, Control, and Dynamics 30(2), Mar.--Apr. 2007.
\item Orlando, G. and Noomen, R., ``Temporary Capture Method based on Energy Surface Reconstruction, Manifolds Dynamics and Patterns Recognition,'' to be published.
\item Orlando, G., Trajectory Optimization for a Mission to Neptune and Triton: release of a Triton orbiter and capture at Triton, Master’s thesis, Delft University of Technology, Sept. 2008.
\item Szebehely, V., Theory of Orbits, Academic Press, 1967.
\item Goldberg, D., Genetic Algorithms in Search, Optimization, and Machine Learning, Addison-Wesley, Reading, 1989.
\item Michalewicz, Z., Genetic Algorithms + Data Structures = Evolution Programs, Springer, Berlin, 3rd ed., 1996.
\item Broyden, C., ``The Convergence of a Class of Double-Rank Minimization Algorithms,'' Journal of the Institute of Mathematics and Applications 6, 1970, pp. 76--90.
\item Fletcher, R., ``A New Approach to Variable Metric Algorithms,'' The Computer Journal 13, 1970, pp. 317--322.
\item Goldfarb, D., ``A Family of Variable Metric Updates Derived by Variational Means,'' Mathematics of Computation 24, 1970, pp. 23--26.
\item Shanno, D., ``Conditioning of Quasi-Newton Methods for Function Minimization,'' Mathematics of Computation 24, 1970, pp. 647--656.
\item Orlando, G., Mooij, E., and Noomen, R., ``Optimal Orbital Stability around Planetary Satellites as Optimization Problem,'' to be published.
\end{enumerate}

\subsection*{Landis}
\begin{enumerate}
\item Oleson (2017)
\item S.R. Oleson et al., COMPASS Final Report: Triton Hopper, NASA CD-2016-127, Final Report to the NASA Innovative Advanced Concepts Program, 2016.
\item S.R. Oleson and G.A. Landis, ``Triton Hopper: Exploring Neptune’s Captured Kuiper Belt Object,'' Planetary Science Vision 2050 Workshop, Washington, DC, Feb. 2017.
\item G.A. Landis, S.R. Oleson, and the COMPASS team, ``A Hopper for Exploring Neptune’s Moon Triton,'' 15th NASA Small Bodies Assessment Group (SBAG) Meeting, JHU/APL, Laurel, MD, June 2016.
\item D.P. Cruikshank, ``Triton, Pluto, and Charon,'' in The New Solar System (4th ed., Beatty, Petersen, Chaikin, eds.), Sky Publishing, 1999, pp. 285--296.
\item D.P. Cruikshank (ed.), Neptune and Triton, Univ. of Arizona Press, 1995.
\item B. Bienstock et al., Neptune Orbiter, Probe, and Triton Lander Mission, Progress in Astronautics and Aeronautics Series 224, AIAA, 2008.
\item R. Bailey and M. Noca, ``Mission Trades for Aerocapture at Neptune,'' AIAA-2004-3843, 40th Joint Propulsion Conference, Fort Lauderdale, FL, July 2004.
\item Y. Yamashita, M. Kato, and M. Arakawa, ``Experimental study on the rheological properties of polycrystalline solid nitrogen and methane: Implications for tectonic processes on Triton,'' Icarus 207(2), June 2010, pp. 972--977.
\item J. Machado-Rodriguez and G.A. Landis, ``Analysis of a Radioisotope Thermal Rocket Engine,'' AIAA-2017-1445, AIAA SciTech Forum 2017, Grapevine, TX, Jan. 2017.
\item D. Palac et al., ``Nuclear Systems Kilopower Overview,'' Nuclear and Emerging Technologies for Space 2016, Huntsville, AL, Feb. 2016.
\item M.A. Gibson, S.R. Oleson, D.I. Poston, and P. McClure, ``NASA’s Kilopower Reactor Development and the Path to Higher Power Missions,'' IEEE Aerospace Conference, Big Sky, MT, Mar. 2017.
\end{enumerate}

\subsection*{Capinski}
\begin{enumerate}
\item Gidea (2025)
\item [1] V.I. Arnold. ``Instability of dynamical systems with several degrees of freedom.'' Sov. Math. Doklady 5:581--585 (1964).
\item [2] T. Kapela, M. Mrozek, D. Wilczak, P. Zgliczyński. ``CAPD::DynSys: a flexible C++ toolbox for rigorous numerical analysis of dynamical systems.'' Commun. Nonlinear Sci. Numer. Simul. 101:105578 (2021).
\item [3] S. Bolotin, D. Treschev. ``Unbounded growth of energy in nonautonomous Hamiltonian systems.'' Nonlinearity 12(2):365 (1999).
\item [4] A. Delshams, R. de la Llave, T.M. Seara. ``A geometric approach to the existence of orbits with unbounded energy in generic periodic perturbations of geodesic flows on $T^2$.'' Comm. Math. Phys. 209(2):353--392 (2000).
\item [5] D. Treschev. ``Evolution of slow variables in near-integrable Hamiltonian systems.'' In Progress in Nonlinear Science, Vol. 1 (Nizhny Novgorod, 2001), pp. 166--169. RAS Institute of Applied Physics, Nizhny Novgorod, 2002.
\item [6] J.N. Mather. ``Arnold diffusion. I. Announcement of results.'' J. Math. Sci. (NY) 124(5):5275--5289 (2004).
\item [7] D. Treschev. ``Evolution of slow variables in a priori unstable Hamiltonian systems.'' Nonlinearity 17(5):1803--1841 (2004).
\item [8] A. Delshams, R. de la Llave, T.M. Seara. ``A geometric mechanism for diffusion in Hamiltonian systems overcoming the large gap problem: heuristics and rigorous verification on a model.'' Mem. Amer. Math. Soc. 179(844): viii+141 (2006).
\item [9] A. Delshams, R. de la Llave, T.M. Seara. ``Orbits of unbounded energy in quasi-periodic perturbations of geodesic flows.'' Adv. Math. 202(1):64--188 (2006).
\item [10] G.N. Piftankin. ``Diffusion speed in the Mather problem.'' Dokl. Akad. Nauk 408(6):736--737 (2006).
\item [11] V. Gelfreich, D. Turaev. ``Unbounded energy growth in Hamiltonian systems with a slowly varying parameter.'' Comm. Math. Phys. 283(3):769--794 (2008).
\item [12] A. Delshams, G. Huguet. ``Geography of resonances and Arnold diffusion in a priori unstable Hamiltonian systems.'' Nonlinearity 22(8):1997 (2009).
\item [13] C.-Q. Cheng, J. Yan. ``Arnold diffusion in Hamiltonian systems: a priori unstable case.'' J. Differential Geom. 82(2):229--277 (2009).
\item [14] J.N. Mather. ``Arnold diffusion by variational methods.'' In Essays in Mathematics and its Applications, pp. 271--285. Springer, 2012.
\item [15] V. Kaloshin, K. Zhang. ``Arnold diffusion for smooth convex systems of two and a half degrees of freedom.'' Nonlinearity 28(8):2699--2720 (2015).
\item [16] P. Bernard, V. Kaloshin, K. Zhang. ``Arnold diffusion in arbitrary degrees of freedom and normally hyperbolic invariant cylinders.'' Acta Mathematica 217(1):1--79 (2016).
\item [17] C.-Q. Cheng, J. Xue. ``Variational approach to Arnold diffusion.'' Science China Mathematics 62(11):2103--2130 (2019).
\item [18] D. Treschev. ``Arnold diffusion far from strong resonances in multidimensional a priori unstable Hamiltonian systems.'' Nonlinearity 25(9):2717--2757 (2012).
\item [19] M. Gidea, R. de la Llave. ``Perturbations of geodesic flows by recurrent dynamics.'' J. Eur. Math. Soc. (JEMS) 19(3):905--956 (2017).
\item [20] V. Gelfreich, D. Turaev. ``Arnold diffusion in a priori chaotic symplectic maps.'' Comm. Math. Phys. 353(2):507--547 (2017).
\item [21] M. Gidea, J.-P. Marco. ``Diffusing orbits along chains of cylinders.'' Discrete and Continuous Dynamical Systems A 40(0), 2020.
\item [22] V. Kaloshin, K. Zhang. Arnold Diffusion for Smooth Systems of Two and a Half Degrees of Freedom. Princeton Univ. Press, 2020.
\item [23] M. Gidea, R. de la Llave, T.M. Seara. ``A general mechanism of diffusion in Hamiltonian systems: qualitative results.'' Comm. Pure Appl. Math. 73(1):150--209 (2020).
\item [24] M. Gidea, R. de la Llave, T.M. Seara. ``A general mechanism of instability in Hamiltonian systems: skipping along a normally hyperbolic invariant manifold.'' Discrete Contin. Dyn. Syst. A, 2019.
\item [25] L. Chierchia, G. Gallavotti. ``Drift and diffusion in phase space.'' Ann. Inst. H. Poincaré Phys. Théor. 60(1):1--144 (1994).
\item [26] J. Féjoz, M. Guardia, V. Kaloshin, P. Roldán. ``Kirkwood gaps and diffusion along mean motion resonances in the restricted planar three-body problem.'' J. Eur. Math. Soc. 18(10):2315--2403 (2016).
\item [27] A. Delshams, M. Gidea, P. Roldán. ``Arnold’s mechanism of diffusion in the spatial circular restricted three-body problem: a semi-analytical argument.'' Physica D 334:29--48 (2016).
\item [28] M. Capinski, M. Gidea, R. de la Llave. ``Arnold diffusion in the planar elliptic restricted three-body problem: mechanism and numerical verification.'' Nonlinearity 30(1):329--360 (2017).
\item [29] A. Delshams, V. Kaloshin, A. de la Rosa, T.M. Seara. ``Global instability in the restricted planar elliptic three body problem.'' Comm. Math. Phys. 366(3):1173--1228 (2019).
\end{enumerate}

\subsection*{Gilliam}
\begin{enumerate}
\item Bettinger (2025)
\item [1] Batista Negri, R., and Prado, A.F.B.A., ``Circular Restricted n-Body Problem,'' Journal of Guidance, Control, and Dynamics 45(7), 2022, pp. 1357--1364.
\item [2] Grasset, O., Dougherty, M., Coustenis, A., Bunce, E., Erd, C., Titov, D., Blanc, M., Coates, A., Drossart, P., Fletcher, L., Hussmann, H., Jaumann, R., Krupp, N., Lebreton, J.-P., Prieto-Ballesteros, O., Tortora, P., Tosi, F., Van Hoolst, T., ``JUpiter ICy Moons Explorer (JUICE): An ESA Mission to Orbit Ganymede and to Characterise the Jupiter System,'' Planetary and Space Science 78, 2013, pp. 1--21.
\item [5] JPL, ``Three-Body Periodic Orbits,'' \texttt{<ssd.jpl.nasa.gov/tools/periodic\_orbits.html>}, 2024.
\item [4] Szebehely, V., Theory of Orbits. The Restricted Problem of Three Bodies, Academic Press, 1967.
\item [5] JPL, ``Three-Body Periodic Orbits,'' <ssd.jpl.nasa.gov/tools/periodic_orbits.html>, 2024.
\item [6] Williams, D.R., ``Uranian Satellites Fact Sheet,'' <nssdc.gsfc.nasa.gov/planetary/factsheet/uraniansatfact.html>, 2016 (accessed Dec 1, 2024).
\item [7] Williams, D.R., ``Solar System Small Worlds Fact Sheet,'' <nssdc.gsfc.nasa.gov/planetary/factsheet/galileanfact_table.html>, 2016 (accessed Dec 1, 2024).
\item [8] Negri, R.B., and Prado, A.F., ``Generalizing the Bicircular Restricted Four-Body Problem,'' Journal of Guidance, Control, and Dynamics 43(6), 2020, pp. 1173--1179.
\item [9] Iuliano, J., ``A Solution to the Circular Restricted N Body Problem in Planetary Systems,'' M.S. thesis, California Polytechnic State Univ., 2016.
\item [10] Gauthier, R., ``Dynamical Aspects in (4+1)-Body Problems,'' M.S. thesis, Wilfrid Laurier University, 2023.
\item [11] Wilmer, A., and Bettinger, R., ``Lagrangian Derivation and Stability Analysis of Multi-Body Gravitational Dynamical Models with Application to Cislunar Periodic Orbit Propagation,'' Proc. 2021 AAS/AIAA Astrodynamics Specialist Conference, 2021.
\item [12] Wilmer, A., and Bettinger, R., ``Lagrangian Dynamics and the Discovery of Cislunar Periodic Orbits,'' Nonlinear Dynamics (2023). https://doi.org/10.1007/s11071-022-07829-1.
\end{enumerate}

\subsection*{Miceli et al. 2024}
\begin{enumerate}
\item [1] National Academies of Sciences, Engineering, and Medicine, ``NASA 2023 Decadal Survey,'' Technical Report, National Academies Press, Washington, D.C., 2023.
\item [2] Swenson, B., ``Neptune atmospheric probe mission,'' AIAA Guidance, Navigation and Control Conference, 1992. doi:10.2514/6.1992-4371.
\item [3] Masters, A., et al., ``Neptune and Triton: Essential pieces of the Solar System puzzle,'' Planetary and Space Science 104, 2014, pp. 108--121. doi:10.1016/j.pss.2014.05.008.
\item [4] Rymer, A.M., Runyon, K.D., Clyde, B., Núñez, J.I., Nikoukar, R., Soderlund, K.M., Sayanagi, K., et al., ``Neptune Odyssey: A Flagship Concept for the Exploration of the Neptune--Triton System,'' Planetary Science Journal 2(5), 2021, p. 184. doi:10.3847/PSJ/abf654.
\item [5] Vallado, D.A., Fundamentals of Astrodynamics and Applications, 5th ed., Microcosm Press, New York, 2022.
\item [6] Marley, M. et al., ``Planetary Science Decadal Survey JPL Rapid Mission Architecture Neptune--Triton--KBO Study Final Report,'' NASA JPL Technical Report, 2010.
\item [7] Melman, J., Orlando, G., Safipour, E., Mooij, E., Noomen, R., ``Trajectory Optimization for a Mission to Neptune and Triton,'' AIAA/AAS Astrodynamics Specialist Conference, 2007, Paper AIAA-2008-7366.
\item [8] Koon, W., Lo, M., Marsden, J., Ross, S., Dynamical Systems, the Three-Body Problem, and Space Mission Design, Springer, New York, 2011.
\item [9] Smith, T.R., and Bosanac, N., ``Constructing Motion Primitive Sets to Summarize Periodic Orbit Families and Hyperbolic Invariant Manifolds in a Multi-Body System,'' Celestial Mechanics and Dynamical Astronomy 134(1), 2022, p. 7. doi:10.1007/s10569-022-10063-x.
\item [10] Smith, T.R., and Bosanac, N., ``A Motion Primitive Approach to Trajectory Design in a Multi-Body System,'' Proc. AAS/AIAA Astrodynamics Specialist Conference, 2022.
\item [11] Wolek, A., and Woolsey, C.A., Model-Based Path Planning, Springer, Cham, 2017, pp. 183--206. doi:10.1007/978-3-319-55372-6_9.
\item [12] Frazzoli, E., ``Robust Hybrid Control for Autonomous Vehicle Motion Planning,'' Ph.D. Dissertation, MIT, 2001.
\item [13] Paranjape, A.A., Meier, K.C., Shi, X., Chung, S.J., Hutchinson, S., ``Motion Primitives and 3D Path Planning for Fast Flight Through a Forest,'' Int. J. Robotics Research 34(3), 2015, pp. 357--377. doi:10.1177/0278364914558017.
\item [14] Wang, B., Gong, J., Zhang, R., Chen, H., ``Learning to Segment and Represent Motion Primitives from Driving Data for Motion Planning Applications,'' Proc. 21st IEEE Intelligent Transportation Systems Conf., Maui, 2018, pp. 1408--1414. doi:10.1109/ITSC.2018.8569913.
\item [15] Reng, L., Moeslund, T.B., Granum, E., ``Finding Motion Primitives in Human Body Gestures,'' in Gesture in Human-Computer Interaction and Simulation: 6th International Gesture Workshop (2005), pp. 133--144. doi:10.1007/11678816_16.
\item [16] Majumdar, A., Tedrake, R., ``Funnel Libraries for Real-Time Robust Feedback Motion Planning,'' Int. J. Robotics Research 36(8), 2017, pp. 947--982. doi:10.1177/0278364917712421.
\item [17] Kulic, D., Takano, W., Nakamura, Y., ``Incremental Learning, Clustering and Hierarchy Formation of Whole Body Motion Patterns using Adaptive Hidden Markov Chains,'' Int. J. Robotics Research 27(7), 2008, pp. 761--784.
\item [18] Dermy, O., Paraschos, A., Ewerton, M., Peters, J., Charpillet, F., Ivaldi, S., ``Prediction of Intention during Interaction with iCub with Probabilistic Movement Primitives,'' Frontiers in Robotics and AI 4, 2017.
\item [19] Clever, D., Harant, M., Koch, H., Mombaur, K., Endres, D., ``Generation of Complex Humanoid Walking Sequences Based on Optimal Control and Learning of Movement Primitives,'' Robotics and Autonomous Systems 83, 2016, pp. 287--298.
\item [20] Cochrane, C.J., Persinger, R.R., Vance, S.D., et al., ``Single- and Multi-Pass Magnetometric Subsurface Ocean Detection and Characterization in Icy Worlds Using PCA: Application to Triton,'' Earth and Space Science 9(2), 2022. doi:10.1029/2021EA002034.
\item [21] Szebehely, V., Theory of Orbits: The Restricted Problem of Three Bodies, Academic Press, London, 1967.
\item [22] NASA, ``Neptunian Satellite Fact Sheet,'' NASA Planetary Fact Sheet, 2023. (<nssdc.gsfc.nasa.gov/planetary/factsheet/neptuniansatfact.html>, accessed 13 Nov 2023).
\item [23] NASA Navigation and Ancillary Information Facility (NAIF), ``NAIF Generic Kernels,'' 2023. (<naif.jpl.nasa.gov/pub/naif/generic_kernels/>).
\item [24] Murray, C.D., and Dermott, S.F., Solar System Dynamics, Cambridge Univ. Press, 1999.
\item [25] Vaquero Escribano, T.M., ``Spacecraft transfer trajectory design exploiting resonant orbits in multi-body environments,'' Ph.D. thesis, Purdue University, 2013.
\item [26] Howell, K.C., ``Three-Dimensional Periodic Halo Orbits,'' Celestial Mechanics 32(1), 1984, p. 53.
\item [27] Conway, B.A., Spacecraft Trajectory Optimization, Cambridge Univ. Press, 2010.
\item [28] Topputo, F., Zhang, C., ``Survey of Direct Transcription for Low-Thrust Trajectory Optimization with Applications,'' Abstract and Applied Analysis (2014), Article ID 851720. doi:10.1155/2014/851720.
\item [29] Betts, J.T., ``Survey of Numerical Methods for Trajectory Optimization,'' Journal of Guidance, Control, and Dynamics 21(2), 1998, pp. 193--207. doi:10.2514/2.4231.
\item [30] Grebow, D.J., Pavlak, T.A., ``MCOLL: Monte Collocation Trajectory Design Tool,'' AAS/AIAA Astrodynamics Specialist Conference, Stevenson, WA, Aug. 2017. (JPL Document ID 2014/46415).
\item [31] Smith, T.R., Bosanac, N., ``Motion Primitive Approach to Spacecraft Trajectory Design in a Multi-body System,'' Journal of the Astronautical Sciences 70(3--4), 2023. doi:10.1007/s40295-023-00395-7.
\item [32] Kelly, M., ``An Introduction to Trajectory Optimization: How to Do Your Own Direct Collocation,'' SIAM Review 59(4), 2017, pp. 849--904. doi:10.1137/16M1062569.
\item [33] Pritchett, R.E., ``Strategies for Low-Thrust Transfer Design Based on Direct Collocation Techniques,'' Ph.D. thesis, Purdue University Graduate School, 2020. doi:10.25394/PGS.12739775.v1.
\item [34] Ozimek, M., Grebow, D., Howell, K., ``A Collocation Approach for Computing Solar Sail Lunar Pole-Sitter Orbits,'' Open Aerospace Engineering Journal 3, 2010, pp. 65--75. doi:10.2174/1874146001003010065.
\item [35] Williams, P., ``Hermite-Legendre-Gauss-Lobatto Direct Transcription in Trajectory Optimization,'' Journal of Guidance, Control, and Dynamics 32(4), 2009, pp. 1392--1395. doi:10.2514/1.42731.
\item [36] De Boor, C., ``Good Approximation by Splines with Variable Knots. II,'' in Proc. Conference on the Numerical Solution of Differential Equations, Dundee, Scotland, 1973.
\item [37] Russell, R., Christiansen, J., ``Adaptive Mesh Selection Strategies for Solving Boundary Value Problems,'' SIAM Journal on Numerical Analysis 15(1), 1978, pp. 59--80.
\item [38] Patrikalakis, N.M., Maekawa, T., Cho, W., Shape Interrogation for CAD and Manufacturing, Springer, New York, 2009.
\item [39] Bosanac, N., ``Data-Driven Summary of Natural Spacecraft Trajectories in the Earth-Moon System,'' AAS/AIAA Astrodynamics Specialist Conference, Big Sky, MT, Aug. 2023.
\item [40] Spear, R.L., Bosanac, N., ``Rapid Trajectory Design in Multi-Body Systems Using Sampling-Based Kinodynamic Planning,'' AAS/AIAA Astrodynamics Specialist Conference, Big Sky, MT, Aug. 2023.
\item [41] van der Laan, M., and Bryan, J., ``A new partitioning around medoids algorithm,'' Journal of Statistical Computation and Simulation 73(8), 2003, pp. 575--584. doi:10.1080/0094965031000136012.
\item [42] Han, J., Kamber, M., Pei, J., Data Mining: Concepts and Techniques, Morgan Kaufmann, Waltham, 2012.
\item [43] Bruchko, K.L., Bosanac, N., ``Rapid Trajectory Design in Multi-Body Systems Using Sampling-Based Kinodynamic Planning,'' AAS/AIAA Astrodynamics Specialist Conference, Big Sky, MT, Aug. 2023.
\item [44] Dijkstra, E.W., ``A note on two problems in connexion with graphs,'' in Edsger W. Dijkstra: His Life, Work, and Legacy, 2022, pp. 287--290.
\item [45] Parrish, N.L., Scheeres, D.J., ``Low-Thrust Trajectory Optimization with Simplified SQP Algorithm,'' AAS/AIAA Astrodynamics Specialist Conference, 2017. (NASA Technical Report 20170007868).
\item [46] Wächter, A., Biegler, L.T., ``On the Implementation of a Primal-Dual Interior Point Filter Line Search Algorithm for Large-Scale Nonlinear Programming,'' Mathematical Programming 106(1), 2006, pp. 25--57. doi:10.1007/s10107-004-0559-y.
\end{enumerate}

\subsection*{Gray 1989}
\begin{enumerate}
\item R.E. Van Allen, R.J. Cesarone, D.L. Gray, ``Voyager 2 Navigation to Uranus and Neptune,'' AIAA/AAS Astrodynamics Conference, Paper AIAA-82-1474, Aug. 1982.
\item R.J. Cesarone, K. Francis, W.J. Kosmann, S.E. Matousek, C.L. Potts, R.W. Ridenoure, ``Mission Design Challenges Posed by the Voyager 2 Neptune Encounter,'' AAS/AIAA Astrodynamics Conference, Paper AAS 87-489, Aug. 1987.
\item D.C. Roth, T.H. Taylor, J.A. Wackley, ``Development of Three-Way Ranging for the Voyager Neptune Encounter,'' AIAA/AAS Astrodynamics Conference, Paper AIAA-88-4265, Aug. 1988.
\item C.L. Potts, K. Francis, S. Matousek, R. Cesarone, D.L. Gray, ``Voyager 2 Neptune Targeting Strategy,'' (to be presented) 27th Aerospace Sciences Meeting, Reno, NV, Jan. 1989.
\item A.B. Sergeyevsky, ``Voyager 2: A Grand Tour of the Giant Planets,'' AAS/AIAA Astrodynamics Conference, Paper AAS 81-187, Aug. 1981.
\item D.L. Gray, R.J. Cesarone, R.E. Van Allen, ``Voyager 2 Uranus and Neptune Targeting,'' AIAA/AAS Astrodynamics Conference, Paper AIAA-82-1476, Aug. 1982.
\item D.L. Gray, A.H. Taylor, R.P. Davis, G.D. Lewis, D.C. Roth, ``Voyager 2 Navigation to Uranus,'' AAS/AIAA Astrodynamics Conference, Paper AAS 85-378, Aug. 1985.
\item T.H. Taylor, R.A. Jacobson, S.P. Synnott, G.D. Lewis, J.E. Riedel, D.C. Roth, ``Orbit Determination for the Voyager II Uranus Encounter,'' AIAA/AAS Astrodynamics Conference, Paper AIAA-86-2112, Aug. 1986.
\item S.P. Synnott, A.J. Donegan, J.E. Riedel, J.A. Stuve, ``Interplanetary Optical Navigation: Voyager Uranus Encounter,'' AIAA/AAS Astrodynamics Conference, Paper AIAA-86-2113, Aug. 1986.
\item D.L. Gray, R.J. Cesarone, C.L. Potts, K. Francis, S.E. Matousek, ``Voyager 2 Uranus Navigation results,'' AIAA/AAS Astrodynamics Conference, Paper AIAA-86-2109, Aug. 1986.
\item W.I. McLaughlin, D.M. Wolff, ``Voyager Flight Engineering: Preparing for Uranus,'' AIAA 23rd Aerospace Sciences Meeting, Paper AIAA-85-0287, Jan. 1985.
\item G. Carlisle, M. Hill, ``Voyager Saturn Encounter Attitude and Articulation Control Experience,'' AAS Rocky Mountain Guidance and Control Conference, Paper 81-042, Feb. 1981.
\item H.P. Marderness, ``Voyager Engineering Improvements for Uranus Encounter,'' AIAA/AAS Astrodynamics Conference, Paper AIAA-86-2110, Aug. 1986.
\item R.P. Laeser, ``Operational Compensation for Effect of Close Titan Flyby on Remainder of Voyager 1 Saturn Near Encounter,'' Proc. Int. Symp. on Spacecraft Flight Dynamics, May 1981, pp. 431--440.
\item R.P. Laeser, ``Voyager--Uranus at Our Doorstep,'' IAF-85-390, 36th IAF Congress, Oct. 1985.
\item R.P. Laeser, ``The Engineering of the Voyager 2 Mission to Uranus,'' IAF-86-??, 37th IAF Congress, Oct. 1986.
\item E.C. Stone, ``Voyager 2 Reprogrammed for New Observations at Saturn,'' Nature, Aug. 20, 1981 (News
\item Views).
\item L. Miller, K. Savary, ``Voyager Flight Engineering Preparations for Neptune Encounter,'' AIAA/AAS Astrodynamics Conference, Paper AIAA-88-4263, Aug. 1988.
\end{enumerate}

\subsection*{Ibrahim et al. 2018}
\begin{enumerate}
\item [1] Szebehely, V.G. (1967) Theory of Orbits: The Restricted Problem of Three Bodies. Academic Press, New York.
\item [2] Curtis, H.D. (2009) Orbital Mechanics for Engineering Students, 2nd ed., Elsevier, Amsterdam.
\item [3] Sharma, R.K., Subba Rao, P.V. (1976) ``Stationary solutions and their characteristic exponents in the restricted three-body problem when the more massive primary is an oblate spheroid.'' Celestial Mechanics 13, 137--149. doi:10.1007/BF01232721.
\item [4] Ibrahim, A.H., Ismail, M.N., Khalil, I.K.H. (2016) ``Studying the libration points of the Sun--Earth--Moon system.'' Int. J. Scientific and Engineering Research 7(10).
\item [5] Ismail, M.N., Khalil, I.K.H., Ibrahim, A.H. (2016) ``The effect of solar radiation pressure on the libration points of the restricted four-body problem.'' Global Journal of Advanced Research 3, 901--906.
\item [6] Archambeau, G., Augros, P., Trélat, E. (2008) ``Eight Lissajous Orbits in the Earth--Moon System.'' (Details not provided in reference list.)
\item [7] Celletti, A., Pucacco, G., Stella, D. (2015) ``Lissajous and Halo Orbits in the Restricted Three-Body Problem.'' Journal of Nonlinear Science 25, 343--370. doi:10.1007/s00332-015-9232-2.
\item [8] Abouelmagd, E.I., El-Shaboury, S.M. (2012) ``Periodic orbits under combined effects of oblateness and radiation in the restricted problem of three bodies.'' Astrophysics and Space Science 341, 331--341.
\end{enumerate}

\subsection*{Fiehler}
\begin{enumerate}
\item Oleson (2004)
\item S.R. Oleson, L. Gefert, J. Schreiber, J. McAdams, ``Sub-Kilowatt Radioisotope Electric Propulsion for Outer Solar System Exploration,'' Forum on Innovative Approaches to Outer Planetary Exploration 2001--2020, LPI, Houston, TX, Feb. 2001.
\item S.R. Oleson, L. Gefert, M. Patterson, J. Schreiber, S. Benson, J. McAdams, P. Ostdiek, ``Outer Planet Exploration with Advanced Radioisotope Electric Propulsion,'' IEPC-2001-0179, 27th Int. Electric Propulsion Conf., Pasadena, CA, Oct. 2001.
\item R.J. Noble, ``Radioisotope Electric Propulsion of Small Payloads for Regular Access to Deep Space,'' AIAA 93-1897, 29th Joint Propulsion Conference, Monterey, CA, 1993.
\item R.J. Noble, ``Radioisotope Electric Propulsion for Small Robotic Space Probes,'' JBIS 49, 455--468 (1996).
\item R.J. Noble, ``Radioisotope Electric Propulsion of Sciencecraft to the Outer Solar System and Near-Interstellar Space,'' Nuclear News, Nov. 1999, pp. 34--40.
\item S.R. Oleson, S. Benson, L. Gefert, M. Patterson, J. Schreiber, ``Radioisotope Electric Propulsion for Fast Outer Planetary Orbiters,'' AIAA-2002-3967, 38th Joint Propulsion Conference, Indianapolis, IN, July 2002.
\item National Research Council, New Frontiers in the Solar System: An Integrated Exploration Strategy, National Academies Press, 2002, Ch. 7.
\item S. Oleson, L. Gefert, S. Benson, M. Patterson, M. Noca, J. Sims, ``Mission Advantages of NEXT: NASA’s Evolutionary Xenon Thruster,'' AIAA-2002-3969, 38th Joint Propulsion Conference, Indianapolis, IN, July 2002.
\item Neptune Orbiter Mission Package, In-Space Propulsion ``Integrated Space Transportation'' Systems Analysis Team report, April 2001.
\item Alagheband, A., Corazzini, T., Duchemin, O., Henny, D., Mason, R., Noca, M., ``Neptune Explorer – An All Solar Powered Neptune Orbiter Mission,'' AIAA-1996-2980, 32nd Joint Propulsion Conference, Lake Buena Vista, FL, July 1996.
\item ``The Vision for Space Exploration,'' NASA, Feb. 2004.
\item J.G. Schreiber, L.G. Thieme, ``Overview of NASA GRC Stirling Technology Development,'' AIAA-2003-6093, 1st Int. Energy Conversion Engineering Conf., Portsmouth, VA, Aug. 2003.
\item E.J. Pencil, H. Kamhawi, L. Arrington, ``Overview of NASA’s Pulsed Plasma Thruster Development Program,'' AIAA-2004-3455, 40th Joint Propulsion Conference, Fort Lauderdale, FL, July 2004.
\item R. Oberto, T. Sweetser, et al., ``Team X Titan Orbiter 2003-10,'' JPL Advanced Projects Design Team Report #658, 2003.
\item M. Patterson, S. Domonkos, J. Foster, T. Haag, ``NEXT: NASA’s Evolutionary Xenon Thruster Development Status,'' AIAA-2003-4862, 39th Joint Propulsion Conf., Huntsville, AL, July 2003.
\item T.K. Phelps, S. Wiseman, D.S. Komm, T. Bond, L. Pinero, ``Development of the NEXT Power Processing Unit,'' AIAA-2003-4867, 39th Joint Propulsion Conf., Huntsville, AL, July 2003.
\item R.S. Aadland, C.S. Engelbrecht, G.B. Ganapathi, D.A. Browning, F. Wilson, W.A. Hoskins, ``Xenon Propellant Management System for 40 cm NEXT Ion Thruster,'' AIAA-2003-4880, 39th Joint Propulsion Conf., Huntsville, AL, July 2003.
\item R. Gershman, ``Propulsion and Power Technology Needs of Future Solar System Exploration Missions,'' Solar System Exploration Theme Overview, March 1999.
\item TRW Space
\item Technology Division, ``Storable Thruster Upgrade Technology Program, Informal Report: Mission Analysis Task,'' for NASA GRC, Sept. 2001.
\item The Boeing Company, Delta IV Payload Planner’s Guide, MDC 00H0043, Oct. 2000.
\item The Boeing Company, Delta IV Payload Planner’s Guide Update, April 2002.
\end{enumerate}

\subsection*{Stuchi et al. 2008}
\begin{enumerate}
\item Brouwer, D., Clemence, M.G. Methods of Celestial Mechanics. Academic Press, New York, 1961.
\item Corrêa, A.A., Ph.D. Thesis, INPE (Brazil), 2005 (in Portuguese).
\item Gómez, G., Jorba, A., Llibre, J., Masdemont, J., Martínez, R., Simó, C., Dynamics and Mission Design Near the Libration Points. World Scientific, 2002.
\item Hammel, H.B., Baines, K.H., Cuzzi, J.D., et al., ``Exploration of the Neptune System,'' ASP Conf. Series 272, 297--323 (2002).
\item Holman, M.J., Kavelaars, J.J., Grav, T., et al., ``Discovery of five irregular moons of Neptune,'' Nature 430, 865--867 (2004).
\item Jorba, A., Masdemont, J., ``Dynamics in the centre manifold of the collinear points of the restricted three-body problem,'' Physica D 132, 189--213 (1999).
\item Jorba, A., ``A methodology for the numerical computation of normal forms, centre manifolds and first integrals of Hamiltonian systems.'' Experimental Mathematics 8, 155--195 (1999).
\item Oberti, P., ``Lagrangian satellites of Thetis and Dione,'' Astronomy
\item Astrophysics 228, 275--283 (1990).
\item Sharma, R.K., Subbarao, P.V., ``Stationary solutions and their characteristic exponents in the restricted three-body problem when the more massive primary is an oblate spheroid,'' Celestial Mechanics 13, 137--149 (1976).
\item Sharma, R.K., ``Periodic orbits of the second kind in the restricted three-body problem when the more massive primary is an oblate spheroid,'' Astrophysics and Space Science 76, 255--258 (1981).
\item Sharma, R.K., ``Periodic orbits of the third kind in the restricted three-body problem with oblateness,'' Astrophysics and Space Science 166, 211--218 (1990).
\item Simó, C., Estabilitat de Sistemes Hamiltonians, Mem. Real Acad. Ciencias Artes Barcelona 48(7), 1989.
\item Simó, C., ``On the analytical and numerical approximation of invariant manifolds,'' in Modern Methods in Celestial Mechanics (Benest
\item Froeschlé, eds.), Ed. Frontières, 1990, pp. 285--330.
\item Simó, C., Stuchi, T.J., ``Stable and unstable manifolds and the destruction of KAM tori in the Hill problem,'' Physica D 140(1), 2000.
\item Solórzano, C.R.H., Sukhanov, A.A., Prado, A.F.B.A., ``Optimization on Transfer to Neptune,'' Proc. ICNPAA Mathematical Problems in Engineering
\item Aerospace, Timișoara, Romania, 2004.
\item Solórzano, C.R.H., Sukhanov, A.A., Prado, A.F.B.A., ``Close Approach to Neptune using Gravity Assists,'' Proc. 56th IAF Congress, Fukuoka, Japan, 2005.
\item Solórzano, C.R.H., Sukhanov, A.A., Prado, A.F.B.A., ``Analysis of trajectories to Neptune using gravity assists,'' Advances in the Astronautical Sciences 144, 447--457 (2006).
\item Stuchi, T.J., ``KAM Tori in the center manifold of the 3-D Hill problem,'' in Advances in Space Dynamics (O.C. Winter, A.F.B.A. Prado, eds.), São José dos Campos, Brazil, 2002, pp. 112--127.
\item Subbarao, P.V., Sharma, R.K., ``On the stability of the triangular points in the restricted three-body problem,'' Astronomy
\item Astrophysics 43, 381--383 (1975).
\item Szebehely, V., Theory of Orbits. Academic Press, New York, 1967.
\item Vieira Neto, E., Winter, O.C., ``Time analysis for temporary gravitational capture: satellites of Uranus,'' Astronomical Journal 122, 440--448 (2001).
\item Yokoyama, T., Santos, M.T., Cardin, G., Winter, O.C., ``On the orbits of the outer satellites of Jupiter,'' Astronomy
\item Astrophysics 401, 763--772 (2003).
\end{enumerate}

\subsection*{Canales et al. 2023}
\begin{enumerate}
\item [1] National Academies of Sciences, Engineering, and Medicine, Origins, Worlds, and Life: A Decadal Strategy for Planetary Science and Astrobiology 2023--2032, National Academies Press, Washington, D.C., 2022, pp. 278--298.
\item [2] Phillips, C.B., Pappalardo, R.T., ``Europa Clipper Mission Concept,'' Eos, Transactions AGU 95(20), 2014, pp. 165--167.
\item [3] Grasset, O., et al., ``JUpiter ICy Moons Explorer (JUICE): An ESA Mission to Orbit Ganymede…,'' Planetary and Space Science 78, Apr. 2013, pp. 1--21.
\item [4] Sims, J.A., ``Jupiter Icy Moons Orbiter Mission Design Overview,'' AAS/AIAA Astrodynamics Specialist Conf., NASA CP-20060043643, 2006.
\item [5] Ross, S.D., Koon, W.S., Lo, M.W., Marsden, J.E., ``Design of a Multi-Moon Orbiter,'' 13th AAS/AIAA Space Flight Mechanics Meeting, AAS 03-143, 2003.
\item [6] Izzo, D., Simoes, L.F., Martens, M., de Croon, G.C., Heritier, A., Yam, C., ``Search for a Grand Tour of the Jupiter Galilean Moons,'' Proc. 15th Genetic and Evolutionary Computation Conf. (GECCO), 2013, pp. 1301--1308.
\item [7] Colasurdo, G., Zavoli, A., Longo, A., Casalino, L., Simeoni, F., ``Tour of Jupiter Galilean Moons: Winning Solution of GTOC6,'' Acta Astronautica 102, Sept. 2014, pp. 190--199.
\item [8] Gómez, G., Koon, W.S., Lo, M.W., Marsden, J.E., Masdemont, J.J., Ross, S.D., ``Invariant Manifolds, the Spatial Three-Body Problem and Space Mission Design,'' Advances in the Astronautical Sciences 109, 2001, pp. 3--22.
\item [9] Gómez, G., Koon, W.S., Lo, M.W., Marsden, J.E., Masdemont, J.J., Ross, S.D., ``Invariant Manifolds, the Spatial Three-Body Problem and Space Mission Design,'' Nonlinearity 17(5), 2004, p. 1571.
\item [10] Koon, W.S., Lo, M.W., Marsden, J.E., Ross, S.D., ``Heteroclinic Connections Between Periodic Orbits and Resonance Transitions in Celestial Mechanics,'' Chaos 10(2), 2000, pp. 427--469.
\item [11] Koon, W.S., Lo, M.W., Marsden, J.E., Ross, S.D., ``Constructing a Low Energy Transfer Between Jovian Moons,'' Celestial Mechanics (AMS Conference Series), Providence, RI, 2002, p. 129.
\item [12] Koon, W.S., Lo, M.W., Marsden, J.E., Ross, S.D., Dynamical Systems, the Three-Body Problem and Space Mission Design, Marsden Books, Wellington, NZ, 2011, pp. 1167--1181.
\item [13] Grover, P., Ross, S., ``Designing Trajectories in a Planet--Moon Environment Using the Controlled Keplerian Map,'' Journal of Guidance, Control, and Dynamics 32(2), 2009, pp. 437--444.
\item [14] Campagnola, S., Russell, R.P., ``Endgame Problem Part 1: $V_\infty$-Leveraging Technique and the Leveraging Graph,'' Journal of Guidance, Control, and Dynamics 33(2), 2010, pp. 463--475.
\item [15] Campagnola, S., Russell, R.P., ``Endgame Problem Part 2: Multi-body Technique and the Tisserand--Poincaré Graph,'' Journal of Guidance, Control, and Dynamics 33(2), 2010, pp. 476--486.
\item [16] Lantoine, G., Russell, R.P., Campagnola, S., ``Optimization of Low-Energy Resonant Hopping Transfers Between Planetary Moons,'' Acta Astronautica 68(7), 2011, pp. 1361--1378.
\item [17] Lantoine, G., Russell, R.P., ``Near Ballistic Halo-to-Halo Transfers Between Planetary Moons,'' Journal of the Astronautical Sciences 58(3), 2011, pp. 335--363.
\item [18] Fantino, E., Castelli, R., ``Efficient Design of Direct Low-Energy Transfers in Multi-Moon Systems,'' Celestial Mechanics and Dynamical Astronomy 127(4), 2017, pp. 429--450.
\item [19] Fantino, E., Flores, R.M., Al-Khateeb, A.N., ``Efficient Two-Body Approximations of Impulsive Transfers Between Halo Orbits,'' Proc. 69th IAC, Paper IAC-18,C1,1,7,x44277, Oct. 2018.
\item [20] Canales, D., Howell, K.C., Fantino, E., ``Moon-to-Moon Transfer Methodology for Multi-Moon Systems in the Coupled Spatial CR3BP,'' Proc. AAS/AIAA Astrodynamics Specialist Conf., AAS 20-462, Aug. 2020.
\item [21] Canales, D., Howell, K.C., Fantino, E., ``Transfer Design Between Neighborhoods of Planetary Moons in the CR3BP: The Moon-to-Moon Analytical Transfer Method,'' Celestial Mechanics and Dynamical Astronomy 133(8), 2021, p. 36.
\item [22] Canales, D., Howell, K.C., Fantino, E., ``A Versatile Moon-to-Moon Transfer Design Method for Applications Involving Libration Point Orbits,'' Acta Astronautica 198, Sept. 2022, pp. 388--402.
\item [23] Canales, D., Gupta, M., Park, B., Howell, K.C., ``Exploration of Deimos and Phobos Leveraging Resonant Orbits,'' Proc. 31st AAS/AIAA Space Flight Mechanics Meeting, AAS 21-234, 2021.
\item [24] Canales, D., Gupta, M., Park, B., Howell, K.C., ``A Transfer Trajectory Framework for the Exploration of Phobos and Deimos Leveraging Resonant Orbits,'' Acta Astronautica 194, May 2022, pp. 263--276.
\item [25] Lara, M., Russell, R.P., Villac, B.F., ``Fast Estimation of Stable Regions in Real Models,'' Meccanica 42, 2007, pp. 511--515.
\item [26] Villac, B.F., ``Using FLI Maps for Preliminary Spacecraft Trajectory Design in Multi-Body Environments,'' Celestial Mechanics and Dynamical Astronomy 102, Sept. 2008, pp. 29--48.
\item [27] Haller, G., Sapsis, T., ``Lagrangian Coherent Structures and the Smallest Finite-Time Lyapunov Exponent,'' Chaos 21(2), 2011, Paper 023115.
\item [28] Canales, D., Howell, K.C., Fantino, E., ``Using Finite-Time Lyapunov Exponent Maps for Planetary Moon-Tour Design,'' Proc. 2021 AAS/AIAA Astrodynamics Specialist Conf., AAS 21-625, 2021.
\item [29] Canales, D., Howell, K.C., Fantino, E., ``Using Finite-Time Lyapunov Exponent Maps for Planetary Moon-Tour Design,'' Proc. AIAA ASCEND 2021, Paper AIAA 2021-4154, 2021.
\item [30] Poincaré, H., Les Méthodes Nouvelles de la Mécanique Céleste, Gauthier-Villars, Paris, 1892, pp. 128--130.
\item [31] Szebehely, V., The General and Restricted Problems of Three Bodies, Springer, Vienna, 1974, pp. 12--49.
\item [32] Broucke, R.A., ``Periodic Orbits in the Restricted Three-Body Problem with Earth--Moon Masses,'' JPL Technical Report 32-1168, Pasadena, 1968.
\item [33] Howell, K.C., Campbell, E.T., ``Three-Dimensional Periodic Solutions that Bifurcate from Halo Families in the CR3BP,'' Advances in the Astronautical Sciences 102, 1999, pp. 891--910.
\item [34] Haapala, A., Howell, K.C., ``A Framework for Constructing Transfers Linking Periodic Libration Point Orbits in the Spatial CR3BP,'' Int. Journal of Bifurcation and Chaos 26(5), 2016, 1630013.
\item [35] Wiggins, S., Guckenheimer, J., ``Chaotic Transport in Dynamical Systems,'' Physics Today 45(7), 1992, pp. 68--69.
\item [36] Gawlik, E.S., Marsden, J.E., Du Toit, P.C., Campagnola, S., ``Lagrangian Coherent Structures in the Planar Elliptic Restricted Three-Body Problem,'' Celestial Mechanics and Dynamical Astronomy 103(3), 2009, pp. 227--249.
\item [37] Short, C.R., Howell, K.C., ``Lagrangian Coherent Structures in Various Map Representations for Application to Multi-Body Gravitational Regimes,'' Acta Astronautica 94(2), 2014, pp. 592--607.
\item [38] Perez-Palau, D., Gómez, G., Masdemont, J., ``Detecting Invariant Manifolds Using Hyperbolic Lagrangian Coherent Structures,'' Proc. 1st IAA Conference on Dynamics and Control of Spacecraft, IAA-AAS-DyCoSS1-08-06, 2012.
\item [39] Haller, G., ``A Variational Theory of Hyperbolic Lagrangian Coherent Structures,'' Physica D 240(7), 2011, pp. 574--598.
\item [40] Huilgol, R., Phan-Thien, N., Kinematics of Fluid Flow (Ch. 1 in Fluid Mechanics of Viscoelasticity, Vol. 6 of Rheology Series), Elsevier, 1997, pp. 1--83.
\item [41] Golub, G., Kahan, W., ``Calculating the Singular Values and Pseudo-Inverse of a Matrix,'' SIAM J. Numerical Analysis 2(2), 1965, pp. 205--224.
\item [42] Smith, D.R., An Introduction to Continuum Mechanics—After Truesdell and Noll, Springer, 1993, pp. 143--162.
\item [43] Short, C.R., Blazevski, D., Howell, K.C., Haller, G., ``Stretching in Phase Space and Applications in General Non-Autonomous Multi-Body Problems,'' Celestial Mechanics and Dynamical Astronomy 122(3), 2015, pp. 213--238.
\item [44] ``SPICE – An Observation Geometry System for Space Science Missions,'' 2023, <naif.jpl.nasa.gov/naif> (retrieved 12 May 2023).
\item [45] Canales, D., ``Transfer Design Methodology Between Neighborhoods of Planetary Moons in the CR3BP,'' Ph.D. Thesis, Purdue University, 2021.
\end{enumerate}

\subsection*{Gawlik 2007}
\begin{enumerate}
\item 1. Dellnitz, M., et al., ``Transport in dynamical astronomy and multibody problems,'' Int. J. Bifurcation and Chaos 15, 699--727 (2005).
\item 2. Porter, M.A., Cvitanović, P., ``Ground control to Niels Bohr: exploring outer space with atomic physics,'' Notices of the AMS 52, 1020--1025 (2005).
\item 3. Shadden, S.C., Lekien, F., Marsden, J.E., ``Definition and properties of Lagrangian coherent structures from finite-time Lyapunov exponents in two-dimensional aperiodic flows,'' Physica D 212, 271--304 (2005).
\item 4. Koon, W.S., Lo, M., Marsden, J.E., Ross, S.D., ``Heteroclinic connections between periodic orbits and resonance transitions in celestial mechanics,'' Chaos 10, 427--469 (2000).
\item 5. Marsden, J.E., Ross, S.D., ``New methods in celestial mechanics and mission design,'' Bulletin of the AMS 43, 43--73 (2005).
\item 6. Koon, W.S., Lo, M.W., Marsden, J.E., Ross, S.D., ``Resonance and capture of Jupiter comets,'' Celestial Mechanics and Dynamical Astronomy 81, 27--38 (2001).
\item 7. Gómez, G., Koon, W.S., Lo, M.W., Marsden, J.E., Masdemont, J., Ross, S.D., ``Connecting orbits and invariant manifolds in the spatial restricted three-body problem,'' Nonlinearity 17, 1571--1606 (2004).
\item 8. Koon, W.S., Lo, M.W., Marsden, J.E., Ross, S.D., ``Dynamical systems, the three-body problem, and space mission design,'' in Differential Equations and Dynamical Systems (World Scientific, 2000), pp. 1167--1181.
\item 9. Parker, T.S., Chua, L.O., Practical Numerical Algorithms for Chaotic Systems, Springer-Verlag, New York, 1989.
\item 10. Szebehely, V.G., Theory of Orbits: The Restricted Problem of Three Bodies, Academic Press, New York, 1967.
\item 11. Goldstein, H., Poole, C., Safko, J., Classical Mechanics, Addison Wesley, San Francisco, 2002.
\item 12. Ross, S.D., ``Cylindrical Manifolds and Tube Dynamics in the Restricted Three-Body Problem,'' Ph.D. thesis, Caltech, 2004.
\item 13. Lekien, F., Shadden, S.C., Marsden, J.E., ``Lagrangian coherent structures in n-dimensional systems,'' Journal of Mathematical Physics 48, 017504 (2007).
\item 14. Arnold, V.I., Mathematical Methods of Classical Mechanics, Springer, New York, 1989.
\item 15. Marsden, J.E., West, M., ``Discrete mechanics and variational integrators,'' Acta Numerica 10, 357--514 (2001).
\item 16. Press, W.H., Teukolsky, S.A., Vetterling, W.T., Flannery, B.P., Numerical Recipes in C: The Art of Scientific Computing, Cambridge Univ. Press, 1992.
\item 17. Piegl, L., Tiller, W., The NURBS Book, Springer-Verlag, New York, 1997.
\end{enumerate}

\subsection*{Munoz-Gutierrez et al. 2025}
\begin{enumerate}
\item Alexandersen, M., Gladman, B., Kavelaars, J.J., et al. 2016, AJ 152, 111. doi:10.3847/0004-6256/152/5/111.
\item Balaji, S., Zaveri, N., Hayashi, N., et al. 2023, MNRAS 524, 3039. doi:10.1093/mnras/stad2026.
\item Bannister, M.T., Gladman, B.J., Kavelaars, J.J., et al. 2018, ApJS 236, 18. doi:10.3847/1538-4365/aab77a.
\item Barr, A.C., Schwamb, M.E. 2016, MNRAS 460, 1542. doi:10.1093/mnras/stw1052.
\item Chen, Y.-T., Gladman, B., Volk, K., et al. 2019, AJ 158, 214. doi:10.3847/1538-3881/ab480b.
\item Chiang, E.I., Jordan, A.B. 2002, AJ 124, 3430. doi:10.1086/344605.
\item Dias-Oliveira, A., Sicardy, B., Ortiz, J.L., et al. 2017, AJ 154, 22. doi:10.3847/1538-3881/aa74e9.
\item Fernández, J.A., Ip, W.H. 1984, Icarus 58, 109. doi:10.1016/0019-1035(84)90101-5.
\item Forgács-Dajka, E., Kővári, E., Kovács, T., Kiss, C., Sándor, Z. 2023, arXiv:2302.01221. doi:10.48550/arXiv.2302.01221.
\item Forgács-Dajka, E., Sándor, Z., Érdi, B. 2018, MNRAS 477, 3383. doi:10.1093/mnras/sty641.
\item Gladman, B., Lawler, S.M., Petit, J.-M., et al. 2012, AJ 144, 23. doi:10.1088/0004-6256/144/1/23.
\item Gomes, R.S. 2000, AJ 120, 2695. doi:10.1086/316816.
\item Hahn, J.M., Malhotra, R. 2005, AJ 130, 2392. doi:10.1086/452638.
\item Harris, C.R., Millman, K.J., van der Walt, S.J., et al. 2020, Nature 585, 357. doi:10.1038/s41586-020-2649-2.
\item Hunter, J.D. 2007, Computing in Science
\item Engineering 9, 90. doi:10.1109/MCSE.2007.55.
\item Ida, S., Bryden, G., Lin, D.N.C., Tanaka, H. 2000, ApJ 534, 428. doi:10.1086/308720.
\item Ip, W.H., Fernández, J.A. 1997, A
\item A 324, 778.
\item Lellouch, E., Santos-Sanz, P., Lacerda, P., et al. 2013, A
\item A 557, A60. doi:10.1051/0004-6361/201322047.
\item Levison, H.F., Morbidelli, A., Van Laerhoven, C., Gomes, R., Tsiganis, K. 2008, Icarus 196, 258. doi:10.1016/j.icarus.2007.11.035.
\item Li, H., Zhou, L.-Y. 2023, A
\item A 680, A68. doi:10.1051/0004-6361/202346636.
\item Lykawka, P.S., Mukai, T. 2007, Icarus 189, 213. doi:10.1016/j.icarus.2007.01.001.
\item Malhotra, R. 1993, Nature 365, 819. doi:10.1038/365819a0.
\item Malhotra, R. 1995, AJ 110, 420. doi:10.1086/117532.
\item Malhotra, R. 1998, Lunar and Planetary Science Conference, Abstract 1476.
\item Malhotra, R., Chen, Z. 2023, MNRAS 521, 1253. doi:10.1093/mnras/stad483.
\item Milani, A., Nobili, A.M., Carpino, M. 1989, Icarus 82, 200. doi:10.1016/0019-1035(89)90031-6.
\item Mommert, M., Harris, A.W., Kiss, C., et al. 2012, A
\item A 541, A93. doi:10.1051/0004-6361/201118562.
\item Morbidelli, A. 1997, Icarus 127, 1. doi:10.1006/icar.1997.5681.
\item Muñoz-Gutiérrez, M.A., Peimbert, A., Lehner, M.J., Wang, S.-Y. 2021, AJ 162, 164. doi:10.3847/1538-3881/ac1102.
\item Muñoz-Gutiérrez, M.A., Peimbert, A., Pichardo, B. 2018, AJ 156, 108. doi:10.3847/1538-3881/aad4f8.
\item Muñoz-Gutiérrez, M.A., Peimbert, A., Pichardo, B., Lehner, M.J., Wang, S.-Y. 2019, AJ 158, 184. doi:10.3847/1538-3881/ab4399.
\item Muñoz-Gutiérrez, M.A., Pichardo, B., Reyes-Ruiz, M., Peimbert, A. 2015, ApJL 811, L21. doi:10.1088/2041-8205/811/2/L21.
\item Murray, C.D., Dermott, S.F. 1999, Solar System Dynamics. Cambridge Univ. Press.
\item Murray-Clay, R.A., Chiang, E.I. 2005, ApJ 619, 623. doi:10.1086/426425.
\item Nesvorný, D. 2018, ARA
\item A 56, 137. doi:10.1146/annurev-astro-081817-052028.
\item Nesvorný, D., Roig, F., Ferraz-Mello, S. 2000, AJ 119, 953. doi:10.1086/301208.
\item Petit, J.-M., Kavelaars, J.J., Gladman, B.J., et al. 2011, AJ 142, 131. doi:10.1088/0004-6256/142/4/131.
\item Rein, H., Liu, S.F. 2012, A
\item A 537, A128. doi:10.1051/0004-6361/201118085.
\item Rein, H., Spiegel, D.S. 2015, MNRAS 446, 1424. doi:10.1093/mnras/stu2164.
\item Rein, H., Hernandez, D.M., Tamayo, D., et al. 2019, MNRAS 485, 5490. doi:10.1093/mnras/stz769.
\item Robutel, P., Laskar, J. 2001, Icarus 152, 4. doi:10.1006/icar.2000.6576.
\item Stern, S.A., Grundy, W.M., McKinnon, W.B., Weaver, H.A., Young, L.A. 2018, ARA
\item A 56, 357. doi:10.1146/annurev-astro-081817-051935.
\item Tiscareno, M.S., Malhotra, R. 2009, AJ 138, 827. doi:10.1088/0004-6256/138/3/827.
\item Vilenius, E., Kiss, C., Müller, T., et al. 2014, A
\item A 564, A35. doi:10.1051/0004-6361/201322416.
\item Volk, K., Murray-Clay, R., Gladman, B., et al. 2016, AJ 152, 23. doi:10.3847/0004-6256/152/1/23.
\item Wyatt, M.C. 2003, ApJ 598, 1321. doi:10.1086/379064.
\end{enumerate}

\subsection*{Bury}
\begin{enumerate}
\item McMahon (2020)
\item [1] Lo, M.W., Ross, S.D., ``Low Energy Interplanetary Transfers Using Invariant Manifolds of $L_1$, $L_2$, and Halo Orbits,'' NASA Tech Brief 23, 1999.
\item [2] Gómez, G., Koon, W.S., Lo, M.W., Marsden, J.E., Masdemont, J., Ross, S.D., ``Connecting Orbits and Invariant Manifolds in the Spatial Restricted Three-Body Problem,'' Nonlinearity 17, 1571--1606 (2004).
\item [3] Koon, W.S., Lo, M.W., Marsden, J.E., Ross, S.D., Dynamical Systems, the Three-Body Problem, and Space Mission Design, 2006 (available online).
\item [4] Parker, J.S., ``Low-Energy Ballistic Lunar Transfers,'' Ph.D. thesis, Univ. of Colorado Boulder, 2007.
\item [5] Vaquero Escribano, T.M., ``Spacecraft Transfer Trajectory Design Exploiting Resonant Orbits in Multi-body Environments,'' Ph.D. thesis, Purdue Univ., 2013.
\item [6] von Kirchbach, C., Zheng, H., Aristoff, J., Kavanagh, J., Villac, B., Lo, M., ``Trajectories Leaving a Sphere in the Restricted 3-Body Problem,'' Advances in the Astronautical Sciences 120(II), 1875--1901 (2005).
\item [7] Baoyin, H., McInnes, C.R., ``Trajectories to and from the Lagrange Points and the Primary Body Surfaces,'' Journal of Guidance, Control, and Dynamics 29(4), 998--1003 (2006).
\item [8] Alessi, E.M., Gómez, G., Masdemont, J.J., ``Leaving the Moon by Means of Invariant Manifolds of Libration Point Orbits,'' Commun. Nonlinear Sci. Numer. Simul. 14(12), 4153--4167 (2009).
\item [9] Davis, K.E., Anderson, R.L., Scheeres, D.J., Born, G.H., ``The Use of Invariant Manifolds for Transfers Between Unstable Periodic Orbits of Different Energies,'' Celestial Mechanics and Dynamical Astronomy 107(4), 471--485 (2010).
\item [10] Davis, K.E., Anderson, R.L., Scheeres, D.J., Born, G.H., ``Optimal Transfers Between Unstable Periodic Orbits Using Invariant Manifolds,'' Celestial Mechanics and Dynamical Astronomy 109(3), 241--264 (2011).
\item [11] Anderson, R.L., Parker, J.S., ``Comparison of Low-Energy Lunar Transfer Trajectories to Invariant Manifolds,'' Celestial Mechanics and Dynamical Astronomy 115, 311--331 (2013).
\item [12] Parker, J.S., Anderson, R.L., Simon, M.K., Low-Energy Lunar Trajectory Design, JPL/Caltech, 2013.
\item [13] Anderson, R.L., Lo, M.W., ``Spatial Approaches to Moons from Resonance Relative to Invariant Manifolds,'' Acta Astronautica 105(1), 355--372 (2014).
\item [14] Anderson, R.L., ``Approaching Moons from Resonance via Invariant Manifolds,'' Journal of Guidance, Control, and Dynamics 38(6), 2015.
\item [15] Joffre, E., Zamaro, M., Silva, N., Marcos, A., Simplício, P., ``Trajectory design and guidance for landing on Phobos,'' Acta Astronautica 151, 389--400 (2018).
\item [16] Restrepo, R.L., Russell, R.P., Lo, M.W., ``Europa Lander Trajectory Design Using Lissajous Staging Orbits,'' AAS Astrodynamics Specialist Conference, 2018.
\item [17] Restrepo, R.L., ``Patched Periodic Orbits: A Systematic Strategy for Low-Energy Trajectory and Moon Tour Design,'' Ph.D. thesis, 2018.
\item [18] Davis, D.C., Boudad, K.K., Power, R.J., Howell, K.C., ``Heliocentric Escape and Lunar Impact from Near Rectilinear Halo Orbits,'' AAS Astrodynamics Specialist Conference, Portland, ME, 2019.
\item [19] Roy, A.E., Orbital Motion, 4th ed., IOP Publishing, 2005.
\item [20] Howell, K.C., ``Three-dimensional, periodic ‘halo’ orbits,'' Celestial Mechanics 32(1), 53--71 (1984).
\item [21] Bury, L., McMahon, J., ``Low-Energy Trajectories As Staging Points for Landing on the Secondary Body in the CR3BP,'' AAS Astrodynamics Specialist Conference, Maui, HI, 2019.
\end{enumerate}

\subsection*{Balint 2005}
\begin{enumerate}
\item [1] President G.W. Bush, A Renewed Spirit of Discovery: The President’s Vision for U.S. Space Exploration, Jan. 2004.
\item [2] L.A.M. Benner, The Encyclopedia of Planetary Sciences. Chapman
\item Hall, New York, 1997.
\item [3] T. Balint, Europa Surface Science Package Feasibility Assessment, JPL Technical Report D-30050, Sept. 2004.
\item [4] NASA, ``NASA’s Mars Exploration Program,'' <marsprogram.jpl.nasa.gov>, 2004.
\item [5] T. Balint, N. Emis, ``Thermal Analysis of a Small-RPS Concept for the Mars NetLander Network Mission,'' in M.S. El-Genk (ed.), STAIF-2005, AIP Conf. Proc. 746, 2005.
\item [6] M. Noca, R.W. Bailey, ``Mission Trades for Aerocapture at Neptune,'' AIAA-2004-3843, 40th Joint Propulsion Conference, July 2004.
\item [7] R. Bailey, J. Hall, T. Spilker, N. Okong’o, ``Neptune Aerocapture Mission and Spacecraft Design Overview,'' AIAA-2004-3842, 40th Joint Propulsion Conference, July 2004.
\item [8] NASA-KSC, ``Launch vehicle database,'' <elvperf.ksc.nasa.gov/elvMap/>, July 2004.
\item [9] R. Haw, ``Mass allocation and ΔV requirements for a Triton Lander,'' Personal communication, June 2004.
\item [10] T.S. Balint, ``Small Power System Trade Options for Advanced Mars Mission Studies,'' Proc. 55th IAC (IAC-04-Q.3.b.08), Vancouver, Oct. 2004.
\item [11] P. Anderson, ``SRG110 Program Overview,'' briefing at JPL, June 28, 2004.
\item [12] R. Rovang, MMRTG Preliminary Design Review Data Package, Boeing, Feb. 24, 2004.
\item [13] R.J. Lipinski, S.A. Wright, M.P. Sherman, R.X. Lenard, R.A. Talandis, D.I. Poston, et al., ``Small Fission Power Systems for Mars,'' in M.S. El-Genk (ed.), STAIF-2002, AIP Conf. Proc. 608, 2002, pp. 1043--1053.
\item [14] I. Jun, ``Peer Review for Radiation Shielding Approach,'' JPL Technical Baseline Review Presentation, Feb. 2004.
\end{enumerate}

\subsection*{Neptune–Triton Flagship NASA, 2008}
\begin{enumerate}
\item [1] References not found in the provided sources.
\end{enumerate}

\subsection*{Nakamiya et al. 2012}
\begin{enumerate}
\item [References not extracted in provided sources]
\end{enumerate}

\subsection*{Loeffler et al. 2024}
\begin{enumerate}
\item [References not extracted in provided sources]
\end{enumerate}

\subsection*{Fitzgerald}
\begin{enumerate}
\item Ross (2022)
\item [References not extracted in provided sources]
\end{enumerate}

\subsection*{Saturn Oblateness Halo Orbits 2021}
\begin{enumerate}
\item [References not extracted in provided sources]
\end{enumerate}
\end{document}